\documentclass{article}
\usepackage[scaled]{helvet}
\renewcommand\familydefault{\sfdefault}
\usepackage[T1]{fontenc}
\usepackage[utf8]{inputenc}
\usepackage[english]{babel}
\usepackage{graphicx}
\graphicspath{ {images/} }
\usepackage[all]{nowidow}
\usepackage{textcomp}
\usepackage{outlines}
\usepackage{listings}
\usepackage{spverbatim}
\usepackage{hyperref}

\usepackage[usenames, dvipsnames]{color}
\definecolor{logoblue}{RGB}{12, 27, 42}
\definecolor{logogreen}{RGB}{71, 82, 25}
\definecolor{successgreen}{RGB}{67, 160, 71}
\definecolor{failurered}{RGB}{229, 57, 53}


\setlength{\parindent}{4em}
\setlength{\parskip}{1em}


\widowpenalty10000
\clubpenalty10000

\begin{document}
\title{Creating a PXE Server}
\maketitle
\begin{flushleft}
  \pagebreak
\section{Purpose}
The purpose of this document is to give a baseline instruction set of how to build a PXE and storage server suitable for use with Panucci.
\subsection{End Goals}
\begin{itemize}
  \item PXE Server to boot client machines from
  \item NFS server to serve client root and images
  \item Client image
  \item NAT passthrough for network traffic
\end{itemize}
\subsection{Outline of Path}
\begin{enumerate}
  \item Install Ubuntu Server on server machine
  \item Configure Network Interfaces
  \item Configure DHCP
  \item Configure TFTP
  \item Configure \verb|pxelinux|
  \item Configure NFS
  \item Create client base
  \item Configure client base for network boot
  \item Copy client base to server
  \item Finish configuring \verb|pxelinux|
  \item Test and boot
\end{enumerate}
\
\end{flushleft}
\end{document}
