\documentclass{article}
\usepackage[scaled]{helvet}
\renewcommand\familydefault{\sfdefault}
\usepackage[T1]{fontenc}
\usepackage[utf8]{inputenc}
\usepackage[english]{babel}
\usepackage{outlines}
\usepackage{listings}
\usepackage{spverbatim}
\usepackage{hyperref}


\setlength{\parindent}{4em}
\setlength{\parskip}{1em}

\begin{document}
\title{Using Panucci}
\maketitle
\begin{flushleft}
\section{What is Panucci?}
Panucci is a network-based testing and imaging platform, with a goal of simplifying the workflow of a computer buildout operation to make a single-piece flow viable and productive.
\pagebreak
\section{Intended workflow}
\begin{enumerate}
  \item Gather empty units for orders
  \item Grab a single unit
  \item Install HDD and RAM into unit
  \item Connect unit to server and input/output
  \item Power on unit
  \item Continue building and adding units as unit tests
  \item Upon test completion, begin imaging
  \item After imaging, verify correct boot
  \item Send unit to shipping
\end{enumerate}
\pagebreak
\section{Features}
\begin{itemize}
  \item Single workflow for all units
  \item Built-in partitioning and image creation
  \item Software allows for adding modular hardware
  \item HDD Testing (using \verb|smartctl| and \verb|seeker|)
  \begin{itemize}
    \item Linear Write Speed
    \item Linear Read Speed
    \item SMART Short Self-Test
    \item SMART Health Check
    \item Random Seek Test
  \end{itemize}
  \item RAM testing using \verb|memtester|
  \item Automatic integration with SellerCloud using the \verb|ScrubDeku| gem
  \item Support for label printing
  \item Basic logging, with upcoming support for a database logging system
  \item Used with \verb|Seymour|, a script for creating a full image tree off of one image
  \item Simple interface built on CSS and HTML, allowing easy modification and updating
\end{itemize}
\pagebreak
\section{Basic Use}
This section assumes you have already created a working PXE server for Panucci, and are able to boot into it.  See document "Creating A PXE Server" for information on this.  These instructions assume your existing setup followed the same basic setup as that detailed in "Creating a PXE Server".
\subsection{Starting the server}
The server for Panucci can simply be powered on.  As long as there are no errors, it will automatically begin serving up the needed files.
\subsection{Booting Panucci}
To begin using Panucci, after building out a unit with RAM and HDD, connect it to appropriate network port and power it on.  Assuming a blank drive, network boot should be automatic as long as the unit supports it.
\subsection{Testing in Panucci}
Testing begins automatically on boot.  Users can monitor testing on the displayed status.
\subsection{Order Verification}
Panucci supports order verification using SellerCloud
\end{flushleft}
\end{document}
